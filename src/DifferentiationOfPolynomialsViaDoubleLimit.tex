\documentclass[12pt,letterpaper,oneside,reqno]{amsart}
\usepackage{amsfonts}
\usepackage{amsmath}
\usepackage{amssymb}
\usepackage{amsthm}
\usepackage{float}
\usepackage{mathrsfs}
\usepackage{colonequals}
\usepackage[font=small,labelfont=bf]{caption}
\usepackage[left=1in,right=1in,bottom=1in,top=1in]{geometry}
\usepackage[pdfpagelabels,hyperindex,colorlinks=true,linkcolor=blue,urlcolor=magenta,citecolor=green]{hyperref}
\usepackage{graphicx}
\linespread{1.7}
\emergencystretch=1em
\usepackage{array}
\usepackage{etoolbox}
\apptocmd{\sloppy}{\hbadness 10000\relax}{}{}
\raggedbottom

\newcommand \anglePower [2]{\langle #1 \rangle \sp{#2}}
\newcommand \bernoulli [2][B] {{#1}\sb{#2}}
\newcommand \curvePower [2]{\{#1\}\sp{#2}}
\newcommand \coeffA [3][A] {{\mathbf{#1}} \sb{#2,#3}}
\newcommand \polynomialP [4][P]{{\mathbf{#1}}\sp{#2} \sb{#3}(#4)}

% ordinary derivatives
\newcommand \derivative [2] {\frac{d}{d #2} #1}                              % 1 - function; 2 - variable;
\newcommand \pderivative [2] {\frac{\partial #1}{\partial #2}}               % 1 - function; 2 - variable;
\newcommand \qderivative [1] {D_{q} #1}                                      % 1 - function
\newcommand \nqderivative [1] {D_{n,q} #1}                                   % 1 - function
\newcommand \qpowerDerivative [1] {\mathcal{D}_q #1}                         % 1 - function;
\newcommand \finiteDifference [1] {\Delta #1}                                % 1 - function;
\newcommand \pTsDerivative [2] {\frac{\partial #1}{\Delta #2}}               % 1 - function; 2 - variable;

% high order derivatives
\newcommand \derivativeHO [3] {\frac{d^{#3}}{d {#2}^{#3}} #1}                % 1 - function; 2 - variable; 3 - order
\newcommand \pderivativeHO [3]{\frac{\partial^{#3}}{\partial {#2}^{#3}} #1}
\newcommand \qderivativeHO [2] {D_{q}^{#2} #1}                               % 1 - function; 2 - order
\newcommand \qpowerDerivativeHO [2] {\mathcal{D}_{q}^{#2} #1}                % 1 - function; 2 - order
\newcommand \finiteDifferenceHO [2] {\Delta^{#2} #1}                         % 1 - function; 2 - order
\newcommand \pTsDerivativeHO [3] {\frac{\partial^{#3}}{\Delta {#2}^{#3}} #1} % 1 - function; 2 - variable;

% central factorials and related symbols
\newcommand \centralFactorial [2] {#1^{[#2]}}
\newcommand \fallingFactorial [2] {\left(#1 \right)^{\underline{#2}}}
\newcommand{\stirlingii}{\genfrac{\{}{\}}{0pt}{}}
\newcommand{\eulerianNumber}{\genfrac{\langle}{\rangle}{0pt}{}}

% for llceil coeffcient
\newcommand{\nobarfrac}{\genfrac{}{}{0pt}{}}
\def\llceil{\left\lceil\kern-3.5pt\left\lceil}
\def\rrfloor{\right\rfloor\kern-3.5pt\right\rfloor}
\newcommand \llceilCoefficient [3] {\llceil \nobarfrac{#1}{#2} \rrfloor_{#3}}


\newtheorem{thm}{Theorem}[section]
\newtheorem{cor}[thm]{Corollary}
\newtheorem{lem}[thm]{Lemma}
\newtheorem{examp}[thm]{Example}
\newtheorem{conj}[thm]{Conjecture}
\newtheorem{defn}[thm]{Definition}

\numberwithin{equation}{section}

\title[LaTeX Template for Github]
{LaTeX Template for Github}
\author[Petro Kolosov]{Petro Kolosov}
\address{Software Developer, DevOps Engineer}
\email{kolosovp94@gmail.com}
\urladdr{https://kolosovpetro.github.io}
\keywords{
    Keyword1, Keyword2
}
\subjclass[2010]{26E70, 05A30}
\date{\today}
\hypersetup{
    pdftitle={LaTeX Template for Github},
    pdfsubject={
        Your Subject List
    },
    pdfauthor={Petro Kolosov},
    pdfkeywords={
        Your Keywords list
    }
}
\begin{document}
    \begin{abstract}
        Differentiation is process of finding the derivative, or rate of change, of a function.
Derivative itself is defined by the limit of function's change divided by the function's argument change
as change tends to zero.
In particular, for polynomials the function's change is calculated via Binomial expansion.
This manuscript provides another approach to reach polynomial's function change as a limit of certain polynomial identity,
and therefore expressing the derivative of polynomial as double limit.

    \end{abstract}

    \maketitle

    \tableofcontents


    \section{Introduction} \label{sec:introduction}
    Differentiation is process of finding the derivative, or rate of change, of a function.
Derivative of a function $f(x)$ over domain $x$ is defined by the limit of function's change divided
by the function's argument change as change tends to zero, i.e
\begin{equation*}
    \odv{f(x)}{x} = \lim_{h \to 0} \left[ \frac{f(x+h) - f(x)}{h} \right]
\end{equation*}
Given the polynomial function $f(x)=x^n, \; n \in \mathbb{N}$ its derivative expressed as follows
\begin{equation}
    \odv{x^n}{x} = \lim_{h \to 0} \left[ \frac{(x+h)^n - x^n}{h} \right] \label{eq:polynomial_derivative}
\end{equation}
Therefore, the change of polynomial function from the nominator of~\eqref{eq:polynomial_derivative}
is being expressed applying Binomial theorem (citation) so that
\begin{equation*}
(x+h)
    ^n - x^n = \sum_{k=1}^{n} \binom{n}{k} x^{n-k} h^k
\end{equation*}
Hence, arriving to well-known identity
\begin{equation*}
    \odv{x^n}{x} = \lim_{h \to 0} \left[ \frac{1}{h} \sum_{k=1}^{n} \binom{n}{k} x^{n-k} h^k \right] = n x^{n-1}
\end{equation*}
More precisely, consider the case $f(x) = x^5, \; x\in\mathbb{R}$
\begin{equation*}
    \odv{x^5}{x} = \lim_{h \to 0} \left[ \frac{5 h^4 x + 10 h^3 x^2 + 10 h^2 x^3 + 5 h x^4}{h} \right] = 5x^4
\end{equation*}
However, there is another approach to express the polynomial function's change $(x+h)^n - x^n$ using polynomial identity
(citation), that is
\begin{equation*}
    \polynomialP{m}{b}{x} = x^{2m+1}, \quad \mathrm{as} \; b \to x
\end{equation*}
Polynomials $\polynomialP{m}{b}{x}$ are polynomials in $(x,b) \in \mathbb{R}$, for example
\begin{equation*}
    \begin{split}
        \polynomialP{0}{b}{x}
        &=b, \\
        \polynomialP{1}{b}{x}
        &=3 b^2 - 2 b^3 - 3 b x + 3 b^2 x, \\
        \polynomialP{2}{b}{x}
        &=10 b^3 - 15 b^4 + 6 b^5 - 15 b^2 x + 30 b^3 x - 15 b^4 x + 5 b x^2 - 15 b^2 x^2 + 10 b^3 x^2, \\
        \polynomialP{3}{b}{x}
        &=-7 b^2 + 28 b^3 - 70 b^5 + 70 b^6 - 20 b^7 + 7 b x - 42 b^2 x + 175 b^4 x - 210 b^5 x + 70 b^6 x \\
        &+ 14 b x^2 - 140 b^3 x^2 + 210 b^4 x^2 - 84 b^5 x^2 + 35 b^2 x^3 - 70 b^3 x^3 + 35 b^4 x^3
    \end{split}
\end{equation*}
Now we can express the polynomial function's change in terms of $\polynomialP{m}{b}{x}$ for odd power polynomials
as limit
\[
    (x+h)^{2m+1} - x^{2m+1} = \lim_{b \to x+h} \left[ \polynomialP{m}{b}{x+h} - x^{2m+1} \right]
\]
For instance, let be $m=2$ then $x^5$ polynomial function's change is
\begin{equation*}
    \begin{split}
    (x+h)
        ^{5} - x^{5} &= \lim_{b \to x+h} \left[ \polynomialP{2}{b}{x+h} - x^{5} \right] \\
        &= \lim_{b \to x+h} \left[ 5 b^2 x - 15 b x^2 - 15 b^2 x^2 + 10 x^3 + 30 b x^3 + 10 b^2 x^3 - 15 x^4 - 15 b x^4 + 5 x^5 \right] \\
        &= h^5 + 5 h^4 x + 10 h^3 x^2 + 10 h^2 x^3 + 5 h x^4
    \end{split}
\end{equation*}
Therefore, the derivative of odd-power polynomial $x^{2m+1}, \; x\in\mathbb{R}, \; m\in\mathbb{N}$ can be expressed in terms of double limit as follows
\begin{equation}
    \odv{x^{2m+1}}{x} = \lim_{h \to 0} \lim_{b \to x+h} \left[ \frac{\polynomialP{m}{b}{x+h} - x^{2m+1}}{h} \right]\label{eq:odd-power-derivative}
\end{equation}
For example, given $m=1$ and therefore $f(x) = x^3$ we get
\begin{equation*}
    \begin{split}
        \odv{x^3}{x} &= \lim_{h \to 0} \left[ 3 h - 2 h^2 + 6 x - 6 h x - 6 x^2 + \frac{3 x^2}{h} \right. \\
        &- \left.\frac{3 x^3}{h} - 3 (h + x) + 3 h (h + x) + 6 x (h + x)- \frac{3 x (h + x)}{h} + \frac{3 x^2 (h + x)}{h} \right] \\
        &=\lim_{h \to 0} \left[ h^2 + 3 h x + 3 x^2 \right] = 3x^2
    \end{split}
\end{equation*}
Even-powered $2m+2, \; m\geq 0$ polynomials can be differentiated similarly, expressing the function's gain in terms
of limit of the polynomial $\polynomialP{m}{b}{x}$, i.e
\[
    (x+h)^{2m+2} - x^{2m+2} = \lim_{b \to x+h} \left[ x\polynomialP{m}{b}{x+h} + h\polynomialP{m}{b}{x+h}  - x^{2m+2} \right]
\]
Given $m=1$, the gain of even-powered polynomial $x^{2m+2}$ in its extended form is
\begin{equation*}
    \begin{split}
    (x+h)
        ^{2m+2} - x^{2m+2}
        &= 3 h^3 - 2 h^4 + 9 h^2 x - 8 h^3 x + 9 h x^2 - 12 h^2 x^2 + 3 x^3 \\
        &- 8 h x^3 - 3 x^4 - 3 h^2 (h + x) + 3 h^3 (h + x) - 6 h x (h + x) \\
        &+ 9 h^2 x (h + x) - 3 x^2 (h + x) + 9 h x^2 (h + x) + 3 x^3 (h + x)
    \end{split}
\end{equation*}
So that generally speaking of even-powered polynomial $x^{2m+2}, \; m\geq 1$ we can conclude that its derivative
can be expressed as double limit similarly to~\eqref{eq:odd-power-derivative}
\begin{equation*}
    \odv{x^{2m+2}}{x} =\lim_{h \to 0} \lim_{b \to x+h} \left[ \frac{x\polynomialP{m}{b}{x+h} + h\polynomialP{m}{b}{x+h}  - x^{2m+2}}{h} \right]
\end{equation*}



    \section{Conclusions}\label{sec:conclusions}
    Conclusions of your manuscript.

    \bibliographystyle{unsrt}
    \bibliography{DifferentiationOfPolynomialsViaDoubleLimit}
    \noindent \textbf{Version:} \texttt{Local-0.1.0}

\end{document}
