Differentiation is process of finding the derivative, or rate of change, of a function.
Derivative of a function $f(x)$ over domain $x$ is defined by the limit of function's change divided
by the function's argument change as change tends to zero, i.e
\begin{equation*}
    \frac{d}{dx} f(x) = \lim_{h \to 0} \frac{f(x+h) - f(x)}{h}
\end{equation*}
Given the polynomial function $f(x)=x^n, \; n \in \mathbb{N}$ its derivative expressed as follows
\begin{equation}
    \frac{d}{dx} x^n = \lim_{h \to 0} \frac{(x+h)^n - x^n}{h} \label{eq:polynomial_derivative}
\end{equation}
Therefore, the change of polynomial function from the nominator of~\eqref{eq:polynomial_derivative}
is being expressed applying Binomial theorem (citation) so that
\begin{equation*}
(x+h)
    ^n - x^n = \sum_{k=1}^{n} \binom{n}{k} x^{n-k} h^k
\end{equation*}
Hence, arriving to well-known identity
\begin{equation*}
    \frac{d}{dx} x^n = \lim_{h \to 0} \frac{1}{h} \sum_{k=1}^{n} \binom{n}{k} x^{n-k} h^k = n x^{n-1}
\end{equation*}
More precisely, consider the case $f(x) = x^5, \; x\in\mathbb{R}$
\begin{equation*}
    \frac{d}{dx} x^5 = \lim_{h \to 0} \frac{5 h^4 x + 10 h^3 x^2 + 10 h^2 x^3 + 5 h x^4}{h} = 5x^4
\end{equation*}
However, there is another approach to express the polynomial function's change $(x+h)^n - x^n$ using polynomial identity
(citation), that is
\begin{equation*}
    \polynomialP{m}{b}{x} = x^{2m+1}, \quad \mathrm{as} \; b \to x
\end{equation*}
Polynomials $\polynomialP{m}{b}{x}$ are polynomials in $(x,b) \in \mathbb{R}$, for example
\begin{equation*}
    \begin{split}
        \polynomialP{0}{b}{x}
        &=b, \\
        \polynomialP{1}{b}{x}
        &=3 b^2 - 2 b^3 - 3 b x + 3 b^2 x, \\
        \polynomialP{2}{b}{x}
        &=10 b^3 - 15 b^4 + 6 b^5 - 15 b^2 x + 30 b^3 x - 15 b^4 x + 5 b x^2 - 15 b^2 x^2 + 10 b^3 x^2, \\
        \polynomialP{3}{b}{x}
        &=-7 b^2 + 28 b^3 - 70 b^5 + 70 b^6 - 20 b^7 + 7 b x - 42 b^2 x + 175 b^4 x - 210 b^5 x + 70 b^6 x \\
        &+ 14 b x^2 - 140 b^3 x^2 + 210 b^4 x^2 - 84 b^5 x^2 + 35 b^2 x^3 - 70 b^3 x^3 + 35 b^4 x^3
    \end{split}
\end{equation*}
Now we can express the polynomial function's change in terms of $\polynomialP{m}{b}{x}$ for odd power polynomials
as limit
\[
    (x+h)^{2m+1} - x^{2m+1} = \lim_{b \to x+h} \left[ \polynomialP{m}{b}{x+h} - x^{2m+1} \right]
\]
For instance, let be $m=2$ then $x^5$ polynomial function's change is
\begin{equation*}
    \begin{split}
        (x+h)^{5} - x^{5} &= \lim_{b \to x+h} \left[ \polynomialP{2}{b}{x+h} - x^{5} \right] \\
                          &= \lim_{b \to x+h} \left[ 5 b^2 x - 15 b x^2 - 15 b^2 x^2 + 10 x^3 + 30 b x^3 + 10 b^2 x^3 - 15 x^4 - 15 b x^4 + 5 x^5 \right] \\
                          &= h^5 + 5 h^4 x + 10 h^3 x^2 + 10 h^2 x^3 + 5 h x^4
    \end{split}
\end{equation*}
